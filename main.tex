% !TEX encoding = UTF-8 Unicode

\documentclass[twocolumn,10pt,a4j]{ltjsarticle}
\usepackage{kougai}

\title{VRデートアプリの提案}
\author{1932158 小池 周平  指導教員 須田 宇宙 准教授}
\date{ }

\begin{document}

\maketitle

\section{はじめに}


%背景

2015年9月に開催された「国連持続可能な開発サミット」でSDGs持続可能な開発目標が掲げられた.この中に少子化も含まれており,
その要因は,晩婚化の進展\cite{sasaki2012},交際率・婚姻率の減少\cite{naikakufu2019}などとされている.
一方,18〜34歳の結婚意欲の調査結果によると,1987年〜2010年の結婚意欲はほとんど変化していない.
すなわち,「結婚はしたいけれど,良い相手に恵まれない」と思っている率が高いこととなる\cite{naikakufu2019}.
これは,「出会いの場の減少」と「交際への不安」と言い換えることができる.
%問題点
よって,出会ってからデートに進展するまでをサポートできれば,婚姻率上昇に繋げられるのではないかと考えた.
近年では,SNSやマッチングアプリを通じて出会う者が増えている.
しかし,2人きりで会うことに恐怖を感じるものも少なくない\cite{prtimes,yoshimura2020}.
そこで本研究では,VRデートアプリを提案し,将来的な婚姻率上昇に繋がる意識改革の可能性について開発を行うことを目的とする.



%目的
\section{問題の原点}
平均初婚年齢は20年間で夫は2.6歳,妻は3.1歳増加している.

内閣府の結婚意欲に関する調査によると18~34歳の結婚意欲は23年間で80\%以上を維持しており,男女それぞれ5.5ポイントと3.5ポイント程度しか減少していない.
出会いの場の減少により,多くの社会人がマッチングアプリを利用している.
一方,マッチングアプリ自体に懸念を抱いている方は多く,マッチングアプリに関する印象アンケートでは「詐欺や宗教勧誘にあいそう」,「犯罪にあいそう」などの回答が多数派であった\cite{prtimes}.
仮に学校や職場,マッチングアプリで合いそうな人を見つけたとしても2人きりで直接会うことに恐怖を感じるという方は多く存在する\cite{yoshimura2020}.また,既に付き合っていても時間がないためにデートをすることができないという人も存在する.


そこで,これらを全て解決し交際について前向きになる方法として仮想的な空間で短時間で体験できるシステムであれば解決するだろうと考えられる.
以上のことから私はVRデートアプリケーションを提案する.
\section{VRデートアプリについて}
VR空間内でデートするアプリでは個別の問題点を以下のように改善するコンセプトを考案する.
まず,会うのが怖いことについては,VR空間内で擬似的にデートを行うことで改善できる.
次に,時間がないことに対しては,デートの時間を40分〜1時間に設定し,仕事帰りのちょっとした時間でも体験できるようにすることで改善できる.
デート時間の前半に共通の体験をしてもらい,後半はその体験についておしゃべりする時間としてお互いの性格を知る機会を設けることとした.
\begin{figure}[h]
\begin{center}
\includegraphics[clip,width=85mm,height=55mm]{apurinaigazou.png}
\end{center}
 \caption{実際のVR内の様子}
\end{figure}

\section{開発したアプリケーションについて}
本研究で開発したアプリケーションはコンセプトに沿うようUnityで開発した.
UnityのWebVR環境で作成することで,ブラウザからのアクセスを可能とさせた.
VR機器を使用し,ブラウザからアクセスをすれば2DだけでなくVRでの視点に切り替えることもできる.
また,websocketによる視点同期も可能としており,VR内で同じ時間を共有することができる.
さらに本アプリケーションはデートを自動進行化しており,デートの行動の制限し道筋を建てることでデート未経験者が失敗しないように配慮をしている.


\begin{thebibliography}{99}
\bibitem{sasaki2012} 佐々木 尚之: ``不確実な時代の結婚-JGSSライフコース調査による潜在的稼得力の影響の検証'', 「家族社会学研究」,第24号,pp152-164(2012)
\bibitem{naikakufu2019} 内閣府: ``少子化対策の現状'', \url{https://www8.cao.go.jp/shoushi/shoushika/whitepaper/measures/w-2016/28webhonpen/html/b1_s1-1-3.html}, 2019/3/26参照


\bibitem{prtimes}PRTIMES:``マッチングアプリは怖い?危ない目に合った?初めて会うまでの期間は!?徹底調査'',
\url{https://prtimes.jp/main/html/rd/p/000000016.000059676.html},2021/10/4参照
\bibitem{yoshimura2020} 古村 健太郎: ``成人のマッチングアプリ利用の背景-成人のマッチングアプリ利用に関する研究'', 「日本心理学会大会発表論文」,第84号,pc-006(2020)

\end{thebibliography}

\end{document}
